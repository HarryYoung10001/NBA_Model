\documentclass{article}
\usepackage{amsmath, amssymb}

\begin{document}

\section*{Team Financial \& Salary Model}

\subsection*{1. Revenue and Cost Function}
The revenue of Team $i$ at time $t$, denoted as $Rev_{T_i}(t)$, is modeled as:
\begin{equation}
    Rev_{T_i}(t) = \lambda_1 Win_i(t) + \lambda_2 \ln(Win_i(t-1)) + \lambda_3 t + \lambda_4 MS_i + C
\end{equation}
where $MS$ represents the Market Size.

The total cost $Cost_{T_i}(t)$ is defined as:
\begin{equation}
    Cost_{T_i}(t) = \sum Sal_{p,i}(t) + \text{Lux\_Tax}_i(t)
\end{equation}

The Operating Income $O\_InT_i(t)$ is:
\begin{equation}
    O\_InT_i(t) = Rev_{T_i}(t) - Cost_{T_i}(t)
\end{equation}

\subsection*{2. Player Salary \& Risk}
Player salary is proportional to their total score:
\begin{equation}
    Sal_{p,i}(t) = r \cdot \text{Total\_Score}_{p,i}(t)
\end{equation}

The risk factor is modeled as:
\begin{equation}
    Risk(t) = L_1 \frac{\sum Sal_{p,i}}{Rev_{T_i}(t)} + L_2 \text{Var}(A - S)
\end{equation}

\subsection*{3. Salary Cap \& Restrictions}
The Salary Cap is determined by the league's total revenue (sharing 51\% among 30 teams):
\begin{equation}
    Sal\_Cap = \frac{\sum_{i=1}^{30} Rev_{T_i}(t) \times 51\%}{30}
\end{equation}

\textbf{Minimum Salary Floor Restriction:}
\begin{equation}
    \sum Sal_{p,i} \ge 0.9 \cdot Sal\_Cap
\end{equation}

\subsection*{4. Apron Rules (Hard Caps)}
We define two "Apron" thresholds based on the Salary Cap:
\begin{align}
    Ap_1 &= 1.3 \cdot Sal\_Cap \\
    Ap_2 &= 1.5 \cdot Sal\_Cap
\end{align}

\textbf{Penalties:}
\begin{itemize}
    \item If $Rev_{T_i}(t) > Ap_1$: Add a constant penalty $PN$ to the cost.
    \item If $Rev_{T_i}(t) > Ap_2$: Revenue $Rev_{T_i}(t)$ stops increasing (Growth Cap).
\end{itemize}

\subsection*{5. Luxury Tax Mechanism}
The Luxury Tax Line is set at:
\begin{equation}
    Lux\_Line = 1.2 \cdot Sal\_Cap
\end{equation}

The tax bracket width $W$ is adjusted based on the cap growth (benchmarked to 2023-24):
\begin{equation}
    W = 5M \times \frac{Sal\_Cap}{Sal\_Cap_{2023-2024}}
\end{equation}

Let $X$ be the taxable amount (amount exceeding the tax line):
\begin{equation}
    X = \sum Sal_{p,i} - Lux\_Line
\end{equation}
*(Note: Handwritten implies X related to Rev, but logically Luxury Tax is based on Salary. Adjusted here for logical consistency based on "Lux-Tax" context)*.

The progressive luxury tax is calculated as:
\begin{equation}
    Lux\_Tax_{T_i} = \sum_{j=0}^{n} \min(X - jW, W) \times T_j
\end{equation}
where $\min(X - jW, W)$ represents the amount falling into the $j$-th bracket (clamped by width $W$).

\textbf{Tax Rates ($T_j$):}
\begin{table}[h]
\centering
\begin{tabular}{cc}
\hline
\textbf{Range (Excess $X$)} & \textbf{Tax Rate ($T_j$)} \\ \hline
$0 \sim W$ & 1.50 \\
$W \sim 2W$ & 1.75 \\
$2W \sim 3W$ & 2.50 \\
$3W \sim 4W$ & 3.25 \\
$> 4W$ & +0.5 per bracket \\ \hline
\end{tabular}
\end{table}

\end{document}